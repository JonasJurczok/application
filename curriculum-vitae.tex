%%%%%%%%%%%%%%%%%%%%%%%%%%%%%%%%%%%%%%%%%
% "ModernCV" CV and Cover Letter
% LaTeX Template
% Version 1.1 (9/12/12)
%
% This template has been downloaded from:
% http://www.LaTeXTemplates.com
%
% Original author:
% Xavier Danaux (xdanaux@gmail.com)
%
% License:
% CC BY-NC-SA 3.0 (http://creativecommons.org/licenses/by-nc-sa/3.0/)
%
% Important note:
% This template requires the moderncv.cls and .sty files to be in the same 
% directory as this .tex file. These files provide the resume style and themes 
% used for structuring the document.
%
%%%%%%%%%%%%%%%%%%%%%%%%%%%%%%%%%%%%%%%%%

%----------------------------------------------------------------------------------------
%	PACKAGES AND OTHER DOCUMENT CONFIGURATIONS
%----------------------------------------------------------------------------------------

\documentclass[11pt,a4paper,sans]{moderncv} % Font sizes: 10, 11, or 12; paper sizes: a4paper, letterpaper, a5paper, legalpaper, executivepaper or landscape; font families: sans or roman

\usepackage[utf8]{inputenc}

\usepackage[ngerman]{babel}

\moderncvstyle{classic} % CV theme - options include: 'casual' (default), 'classic', 'oldstyle' and 'banking'
\moderncvcolor{blue} % CV color - options include: 'blue' (default), 'orange', 'green', 'red', 'purple', 'grey' and 'black'

\usepackage{lipsum} % Used for inserting dummy 'Lorem ipsum' text into the template

\usepackage[scale=0.75]{geometry} % Reduce document margins
\setlength{\hintscolumnwidth}{4cm} % Uncomment to change the width of the dates column
%\setlength{\makecvtitlenamewidth}{10cm} % For the 'classic' style, uncomment to adjust the width of the space allocated to your name

%----------------------------------------------------------------------------------------
%   NAME AND CONTACT INFORMATION SECTION
%----------------------------------------------------------------------------------------

\firstname{Jonas} % Your first name
\familyname{Jurczok} % Your last name

% All information in this block is optional, comment out any lines you don't need
\title{Curriculum Vitae}
\address{Oberhofer Weg 4}{12209 Berlin}
\mobile{+49 157 856 13 856}
\email{jonasjurczok@gmail.com}

\homepage{www.github.com/jonasjurczok}{} % The first argument is the url for the clickable link, the second argument is the url displayed in the template - this allows special characters to be displayed such as the tilde in this example

%\extrainfo{additional information}

\photo[75pt][0pt]{pictures/profile} % The first bracket is the picture height, the second is the thickness of the frame around the picture (0pt for no frame)
%\quote{"Mein Fulli pusselt." - Willi Schönborn}


\begin{document}

\makecvtitle % Print the CV title

%----------------------------------------------------------------------------------------
%	PERSONAL INFORMATION SECTION
%----------------------------------------------------------------------------------------

\section{Persönliche Angaben}

\cvitem{Geburtsdatum}{22. August 1985}
\cvitem{Staatsangehörigkeit}{deutsch}
\cvitem{Familienstand}{ledig, keine Kinder}

%----------------------------------------------------------------------------------------
%	WORK EXPERIENCE SECTION
%----------------------------------------------------------------------------------------

\section{Berufserfahrung}

\cventry{04/2015 -- Heute}{Onboarding Manager}{}{\textsc{Zalando SE}}{Berlin}{\begin{itemize}
\item Betreuung aller Neueinsteiger für Zalando Technology
\item Entwicklung eines Konzeptes zur Einarbeitung neuer Mitarbeiter
\item Konzeptionierung einer Softwarelösung zur Automation des Onboarding Prozesses
%\item Forschung an einem Konzept zur Testbarkeit von verteilten Systemen
\item Aufbau eines Teams
\item Führen von Bewerbungsgesprächen
\end{itemize}}

%------------------------------------------------

\cventry{06/2013 -- 03/2015}{Interim Techlead}{Leitung eines Subteams}{\textsc{Zalando SE}}{Berlin}{\begin{itemize}
\item Aufbau eines Entwicklungsteams
\item Pflegen und Entwickeln der Teamkultur
\item Konzeption und Implementation verschiedener Softwareprojekte im Team
\item Moderation und Konzeptionierung der Teammeetings
\item Stakeholder management
\item Agile Projektmethoden (Scrum und Kanban)
\end{itemize}}

%------------------------------------------------

\cventry{01/2013 -- 05/2013}{Softwareentiwckler}{}{\textsc{Zalando SE}}{Berlin}{\begin{itemize}
\item Konzeptionierung und Implementation von neuen Anforderungen (Java)
\item Qualitätssicherung
\end{itemize}}

%------------------------------------------------

\cventry{05/2011 -- 12/2012}{Softwareentwickler}{Werkstudent}{\textsc{CosmoCode GmbH}}{Berlin}{\begin{itemize}
\item Konzeptionierung und Implementation von Kundenprojekten in Java und PHP
\item Qualitätssicherung
\item Kundenkommunikation
\item Agile Projektmethoden (Scrum und Kanban)
\end{itemize}}

%------------------------------------------------

\cventry{07/2009 -- 04/2011}{Webentwickler}{Werkstudent}{\textsc{Plista GmbH}}{Berlin}{\begin{itemize}
\item Webentwicklung mit PHP
\item Entwicklung neuer Features für internes Tooling
\item Qualitätssicherung
\end{itemize}}

%------------------------------------------------

\cventry{12/2006 -- 07/2009}{Entwickler}{Azubi}{\textsc{HUP AG}}{Berlin}{\begin{itemize}
\item Entwicklung des Firmenproduktes mit Delphi und PL/SQL
\item Konzeptionierung neuer Features
\item Kundenkommunikation
\item Qualitätssicherung
\end{itemize}}

%------------------------------------------------

%----------------------------------------------------------------------------------------
%	EDUCATION SECTION
%----------------------------------------------------------------------------------------

\section{Ausbildung}

\cventry{10/2009 -- 12/2012}{Bachelor of Science}{Informatik}{Freie Universität}{Berlin}{\textit{abgebrochen}}  % Arguments not required can be left empty
\cventry{08/1998 -- 07/2006}{Abitur}{Katholische Schule Liebfrauen}{}{}{}
\cventry{08/1997 -- 08/1998}{Grundschule}{Katholische Schule Herz-Jesu}{}{}{}
\cventry{08/1992 -- 07/1997}{Grundschule}{Clara Grunwald}{}{}{}

%----------------------------------------------------------------------------------------
%	PROJECTS SECTION
%----------------------------------------------------------------------------------------
\section{Referenzen}

\subsection{Toga}
\cvitem{Art}{Open Source}
\cvitem{Jahr}{2016 -- Heute}
\cvitem{Team}{Willi Schönborn}
\cvitem{Quellcode}{\url{https://github.com/zalando/toga}}
\cvitem{Technologien}{\textsc{Java}, \textsc{JUnit}}
\cvitem{Beschreibung}{Toga ist eine Bibliothek die die Arbeit mit \textsc{JSON} erheblich
vereinfachen soll. Das Ziel ist es das Testen verteilter Systeme zu unterstützen in dem man
aus einer \textsc{JSON-Schema} Definition Testdaten generieren und gegen das gleiche Schema validieren kann.}

\vspace{10pt}

\subsection{Testing Microservices}
\cvitem{Art}{Theorie}
\cvitem{Jahr}{2016 -- Heute}
\cvitem{URL}{\url{https://github.com/zalando/testing-microservices}}
\cvitem{Technologien}{\textsc{Text}, \textsc{Java}, \textsc{Spring boot}}
\cvitem{Beschreibung}{Theoretische Dokumentation über das Testen eines Eventsourcing Systems. Der interessante Aspekt ist, dass die Software ohne Staging system oder live Tests abgetestet werden kann. Die Grundidee ist die einzlenen Services transitiv über Contracts zu testen und damit einen kompletten use case abbilden zu können. Das Projekt enthält auch eine Referenzimplementation in \textsc{Java} mit \textsc{Spring-Boot}}

\vspace{10pt}

\subsection{Onboarding automation}
\cvitem{Art}{Neuentwicklung}
\cvitem{Jahr}{09/2015 -- Heute}
\cvitem{Team}{Henrik Andersen, Laura Iinberg, Cassandra Pope}
\cvitem{Technologien}{\textsc{Java}, \textsc{Spring-Boot}, \textsc{Event Sourcing}, \textsc{AWS}}
\cvitem{Beschreibung}{Komplette Konzeptionierung und Entwicklung einer Software zur Automatisierung des Onboarding Prozesses. Ziel ist es den manuellen Aufwand pro Newbie drastisch zu verringen um dem Team die Möglichkeit zu geben mehr Zeit mit den Menschen zu verbringen.}

\vspace{10pt}

\subsection{Dunning Fees}
\cvitem{Art}{Neuentwicklung}
\cvitem{Jahr}{10/2013}
\cvitem{Team}{Pablo Rojas}
\cvitem{Technologien}{\textsc{Java}, \textsc{Spring}}
\cvitem{Beschreibung}{Implementation eines Mahnprozesses der die Mahnungen aus einem Buchhaltungssystem übernimmt und je nach Typ in verschiedene Zustellkanäle verteilt.}

\vspace{10pt}

\subsection{Refund Rebuild}
\cvitem{Art}{Refactoring}
\cvitem{Jahr}{04/2014 -- 06/2014}
\cvitem{Team}{Willi Schönborn, Sarah Jehn}
\cvitem{Technologien}{\textsc{Java}, \textsc{Spring}}
\cvitem{Beschreibung}{Komplette Neuentwicklung der Gutschriftenlogik von Zalando (manuelle Gutschriften, Retouren, Stornierungen). Die Herausforderung war eine sehr komplexe Logik durch eine einfach zu wartende, dafür aber sehr konfigurierbare Logik zu ersetzen.}

\newpage
%----------------------------------------------------------------------------------------
%	TECHNOLOGIES SECTION
%----------------------------------------------------------------------------------------

\section{Technologien}
Mit den folgenden Technologien arbeite ich täglich bis regelmäßig:
\newline

\cvitem{Sprachen}{\textsc{Java}, \textsc{SQL}, \textsc{Python}, \textsc{PHP}, \textsc{HTML}, \textsc{CSS}, \LaTeX, \textsc{Shell}}
\cvitem{Data Stores}{\textsc{PostgreSQL}, \textsc{Oracle DB}}
\cvitem{Libraries}{\textsc{Guava}, \textsc{SLF4J}}
\cvitem{Frameworks}{\textsc{Spring-Boot}, \textsc{Guice}, \textsc{JPA}, \textsc{Servlet API}}
\cvitem{Testing}{\textsc{Junit}, \textsc{EasyMock}, \textsc{Mockito}}
\cvitem{SCM}{\textsc{Git}, \textsc{Subversion}}
\cvitem{Build Tools}{\textsc{Maven}, \textsc{Gradle}}

%----------------------------------------------------------------------------------------
%	LANGUAGES SECTION
%----------------------------------------------------------------------------------------

\section{Sprachkenntnisse}

\cvitemwithcomment{Deutsch}{Muttersprache}{}
\cvitemwithcomment{English}{Verhandlungssicher}{}

%----------------------------------------------------------------------------------------

\end{document}