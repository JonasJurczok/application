%%%%%%%%%%%%%%%%%%%%%%%%%%%%%%%%%%%%%%%%%
% "ModernCV" CV and Cover Letter
% LaTeX Template
% Version 1.1 (9/12/12)
%
% This template has been downloaded from:
% http://www.LaTeXTemplates.com
%
% Original author:
% Xavier Danaux (xdanaux@gmail.com)
%
% License:
% CC BY-NC-SA 3.0 (http://creativecommons.org/licenses/by-nc-sa/3.0/)
%
% Important note:
% This template requires the moderncv.cls and .sty files to be in the same 
% directory as this .tex file. These files provide the resume style and themes 
% used for structuring the document.
%
%%%%%%%%%%%%%%%%%%%%%%%%%%%%%%%%%%%%%%%%%

%----------------------------------------------------------------------------------------
%	PACKAGES AND OTHER DOCUMENT CONFIGURATIONS
%----------------------------------------------------------------------------------------

\documentclass[11pt,a4paper,sans]{moderncv} % Font sizes: 10, 11, or 12; paper sizes: a4paper, letterpaper, a5paper, legalpaper, executivepaper or landscape; font families: sans or roman

\usepackage[utf8]{inputenc}

\usepackage[ngerman]{babel}

\moderncvstyle{classic} % CV theme - options include: 'casual' (default), 'classic', 'oldstyle' and 'banking'
\moderncvcolor{blue} % CV color - options include: 'blue' (default), 'orange', 'green', 'red', 'purple', 'grey' and 'black'

\usepackage{lipsum} % Used for inserting dummy 'Lorem ipsum' text into the template

\usepackage[scale=0.75]{geometry} % Reduce document margins
\setlength{\hintscolumnwidth}{4cm} % Uncomment to change the width of the dates column
%\setlength{\makecvtitlenamewidth}{10cm} % For the 'classic' style, uncomment to adjust the width of the space allocated to your name

%----------------------------------------------------------------------------------------
%   NAME AND CONTACT INFORMATION SECTION
%----------------------------------------------------------------------------------------

\firstname{Jonas} % Your first name
\familyname{Jurczok} % Your last name

% All information in this block is optional, comment out any lines you don't need
\title{Curriculum Vitae}
\address{In den neuen Gärten 14}{12247 Berlin}
\mobile{+49 157 856 13 856}
\email{jonasjurczok@gmail.com}

\homepage{www.github.com/jonasjurczok}{} % The first argument is the url for the clickable link, the second argument is the url displayed in the template - this allows special characters to be displayed such as the tilde in this example

%\extrainfo{additional information}

\photo[75pt][0pt]{pictures/profile} % The first bracket is the picture height, the second is the thickness of the frame around the picture (0pt for no frame)
%\quote{"Mein Fulli pusselt." - Willi Schönborn}


\begin{document}

\makecvtitle % Print the CV title

%----------------------------------------------------------------------------------------
%	PERSONAL INFORMATION SECTION
%----------------------------------------------------------------------------------------

\section{Personal Details}

\cvitem{Date of birth}{22. August 1985}
\cvitem{Citizenship}{German}

%----------------------------------------------------------------------------------------
%	WORK EXPERIENCE SECTION
%----------------------------------------------------------------------------------------

\section{Experience}

\cventry{09/2016 -- today}{Tech Lead (Teamlead)}{}{\textsc{Notebooksbilliger.de AG}}{Berlin}{\begin{itemize}
\item Administrative responsibility for a small dev team (4 devs)
\item Continue effort on modernizing the existing ESB implementation and clean up old code
\item Introduce and facilitate new ideas regarding architecture, communication, agile principles and working culture.
\end{itemize}}

%------------------------------------------------

\cventry{07/2016 -- 09/2016}{Senior Software Engineer}{}{\textsc{Notebooksbilliger.de AG}}{Berlin}{\begin{itemize}
\item Introducing agile methodologies into an existing team
\item Introducing, developing and maintaining coding standards and best practices
\item Introducing and improving a culture of communication and collaboration
\item Started the cleanup of the existing ESB system (see references for details)
\item Introducing modern frameworks to improve and simplify the basic application design patterns
\end{itemize}}

%------------------------------------------------

\cventry{04/2015 -- 07/2016}{Onboarding Manager}{}{\textsc{Zalando SE}}{Berlin}{\begin{itemize}
\item Coaching and supervision of every new hire at Zalando Technology (30-70 / month)
\item Development of a concept on how to integrate all newcomers into the Zalando Tech culture
\item Creating a concept for automating the onboarding process as much as possible
\item Creating/building the team that is responsible for the whole onboarding experience
\item Performed over 100 job interviews for software engineers who applied to Zalando Tech
\end{itemize}}

%------------------------------------------------

\cventry{06/2013 -- 03/2015}{Interim Techlead}{Leading a subteam}{\textsc{Zalando SE}}{Berlin}{\begin{itemize}
\item Creation and development of a software engineering team
\item Cater for and develop the culture of said team
\item Managing and creating concepts for multiple software projects in the teams domain
\item Stakeholder Management
\item Teaching agile methodologies (Scrum and Kanban)
\end{itemize}}

%------------------------------------------------

\cventry{01/2013 -- 05/2013}{Software Engineer}{}{\textsc{Zalando SE}}{Berlin}{\begin{itemize}
\item Creating concepts and implement projects and requirements for different kinds of projects in the teams domain (Java)
\item Quality Assurance
\end{itemize}}

%------------------------------------------------

\cventry{05/2011 -- 12/2012}{Software Engineer}{Working student}{\textsc{CosmoCode GmbH}}{Berlin}{\begin{itemize}
\item Conceptual design and implementation of different client projects in Java and PHP
\item Quality Assurance
\item Customer communication
\item Agile techniques (Scrum and Kanban)
\end{itemize}}

%------------------------------------------------

\cventry{07/2009 -- 04/2011}{Webdeveloper}{Working student}{\textsc{Plista GmbH}}{Berlin}{\begin{itemize}
\item Webdevelopment with PHP
\item Developing new features for internal tooling
\item Quality Assurance
\end{itemize}}

%------------------------------------------------

\cventry{12/2006 -- 07/2009}{Software Engineer}{Apprentice}{\textsc{HUP AG}}{Berlin}{\begin{itemize}
\item Development on the company software with Delphi and PL/SQL
\item Conceptual design of new features
\item Customer Communication
\item Quality Assurance
\end{itemize}}

%------------------------------------------------

%----------------------------------------------------------------------------------------
%	EDUCATION SECTION
%----------------------------------------------------------------------------------------

\section{Education}

\cventry{10/2009 -- 12/2012}{Bachelor of Science}{Computer Science}{Free University}{Berlin}{\textit{not finished}}  % Arguments not required can be left empty
\cventry{08/1998 -- 07/2006}{University-entrance diploma}{Catholic School Liebfrauen}{}{}{}
\cventry{08/1997 -- 08/1998}{Primary School}{Catholic School Herz-Jesu}{}{}{}
\cventry{08/1992 -- 07/1997}{Primary School}{Clara Grunwald}{}{}{}

%----------------------------------------------------------------------------------------
%	PROJECTS SECTION
%----------------------------------------------------------------------------------------
\section{Projects}

\vspace{5pt}

\subsection{Stock Component}
\cvitem{Type}{Development project}
\cvitem{Time}{03/2017 -- Today}
\cvitem{Team}{Yuna Morgenstern, Vincent Raabe}
\cvitem{Technologies}{\textsc{Java}, \textsc{Spring-Boot}}
\cvitem{Description}{Our task is to create a stock service that is capable
of holding all customer facing stock, create and manage reservations and provide an easy to use API for the rest of the company. Special care has to be taken regarding high availablility and resilience as this is the backbone for the shop.}

\vspace{10pt}

\subsection{ESB modernization}
\cvitem{Type}{Refactoring}
\cvitem{Time}{09/2016 -- Today}
\cvitem{Team}{Batyr Malik, Yuna Morgenstern, Michael Mertins, Vincent Raabe}
\cvitem{Technologies}{\textsc{Java}, \textsc{Spring-Boot}}
\cvitem{Description}{The goal of this project is to replace a legacy ESB implementation with something modern and more fitting to the company landscape. The main steps are identifying the valuable parts of the legacy system and then extracting them into stand alone applications.}

\vspace{10pt}

\subsection{Toga}
\cvitem{Type}{Open Source}
\cvitem{Time}{2016}
\cvitem{Team}{Willi Schönborn}
\cvitem{Source}{\url{https://github.com/zalando-incubator/toga}}
\cvitem{Technologies}{\textsc{Java}, \textsc{JUnit}}
\cvitem{Description}{Toga is a library that is intended to simplify the work with \textsc{JSON} significantly. 
The goal is to support testing of distributed systems by providing contracts in form of \textsc{JSON-Schema} that can be validated and used to generate test data.}

\vspace{10pt}

\subsection{Testing Microservices}
\cvitem{Type}{Theoretical work}
\cvitem{Time}{2016}
\cvitem{URL}{\url{https://github.com/zalando-incubator/testing-microservices}}
\cvitem{Technologies}{\textsc{Text}, \textsc{Java}, \textsc{Spring boot}}
\cvitem{Description}{Theoretical documentation about testing a distributed, eventsourced system. The interesting aspect is that with this system you can test a distributed system without staging or live
system tests. The basic idea is to test the individual services in a transitive way via contracts. This way you can cover whole use cases without running a complete end to end test on a staging environment.}

\vspace{10pt}

\subsection{Onboarding automation}
\cvitem{Type}{New Software}
\cvitem{Time}{09/2015 -- 06/2016}
\cvitem{Team}{Henrik Andersen, Laura-Anniina Inberg, Cassandra Pope}
\cvitem{Technologies}{\textsc{Java}, \textsc{Spring-Boot}, \textsc{Event Sourcing}, \textsc{AWS}}
\cvitem{Description}{Conceptual design and implementation for internal tooling to automate the onboarding process as much as possible. The goal was to reduce the manual work per newstarter as much as possible to give the onboarding team the possibility to be as efficient as possible with their time.}

\vspace{10pt}

\subsection{Dunning Fees}
\cvitem{Type}{New Software}
\cvitem{Time}{10/2013}
\cvitem{Team}{Pablo Rojas}
\cvitem{Technologies}{\textsc{Java}, \textsc{Spring}}
\cvitem{Description}{Implementation of the Zalando Dunning process. The goal was to receive the dunnings from the bookkeeping system and send it to the appropriate distribution channels while keeping the whole system as configurable as possible.}

\vspace{10pt}

\subsection{Refund Rebuild}
\cvitem{Type}{Refactoring}
\cvitem{Time}{04/2014 -- 06/2014}
\cvitem{Team}{Willi Schönborn, Sarah Jehn}
\cvitem{Technologies}{\textsc{Java}, \textsc{Spring}}
\cvitem{Description}{Complete rewrite of the refund logic of Zalando (manual refunds, returns, cancellations). The callange was to replace a very complex logic with something easy to maintain but still very configurable.}

\newpage
%----------------------------------------------------------------------------------------
%	TECHNOLOGIES SECTION
%----------------------------------------------------------------------------------------

\section{Technologies}
I work almost daily with the following technologies:
\newline

\cvitem{Languages}{\textsc{Java}, \textsc{SQL}, \textsc{Python}, \textsc{Bash}, \textsc{HTML}, \textsc{CSS}, \LaTeX}
\cvitem{Data Stores}{\textsc{PostgreSQL}, \textsc{Oracle DB}, \textsc{MySQL}}
\cvitem{Libraries}{\textsc{Guava}, \textsc{SLF4J}}
\cvitem{Frameworks}{\textsc{Spring-Boot}, \textsc{Guice}, \textsc{JPA}, \textsc{Servlet API}}
\cvitem{Testing}{\textsc{Junit}, \textsc{EasyMock}, \textsc{Mockito}}
\cvitem{SCM}{\textsc{Git}, \textsc{Subversion}}
\cvitem{Build Tools}{\textsc{Maven}, \textsc{Gradle}}

%----------------------------------------------------------------------------------------
%	LANGUAGES SECTION
%----------------------------------------------------------------------------------------

\section{Language Skills}

\cvitemwithcomment{German}{Mother tongue}{}
\cvitemwithcomment{English}{business fluent}{}

%----------------------------------------------------------------------------------------

\end{document}
